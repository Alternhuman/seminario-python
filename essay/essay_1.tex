% !TEX encoding = UTF-8 Unicode
%%%%%%%%%%%%%%%%%%%%%%%%%%%%%%%%%%%%%%%%%
% Simple Sectioned Essay Template
% LaTeX Template
%
% This template has been downloaded from:
% http://www.latextemplates.com
%
% Note:
% The \lipsum[#] commands throughout this template generate dummy text
% to fill the template out. These commands should all be removed when 
% writing essay content.
%
%%%%%%%%%%%%%%%%%%%%%%%%%%%%%%%%%%%%%%%%%

%----------------------------------------------------------------------------------------
%	PACKAGES AND OTHER DOCUMENT CONFIGURATIONS
%----------------------------------------------------------------------------------------

\documentclass[12pt]{article} % Default font size is 12pt, it can be changed here

\usepackage{geometry} % Required to change the page size to A4
\geometry{a4paper} % Set the page size to be A4 as opposed to the default US Letter

\usepackage{graphicx} % Required for including pictures

\usepackage{float} % Allows putting an [H] in \begin{figure} to specify the exact location of the figure
\usepackage{wrapfig} % Allows in-line images such as the example fish picture

\usepackage{lipsum} % Used for inserting dummy 'Lorem ipsum' text into the template

\linespread{1.2} % Line spacing

%\setlength\parindent{0pt} % Uncomment to remove all indentation from paragraphs

\graphicspath{{Pictures/}} % Specifies the directory where pictures are stored

\usepackage{listings}
\usepackage[spanish]{babel}
\selectlanguage{spanish}
\usepackage[utf8]{inputenc}
\usepackage{hyperref}
\begin{document}

%----------------------------------------------------------------------------------------
%	TITLE PAGE
%----------------------------------------------------------------------------------------

\begin{titlepage}

\newcommand{\HRule}{\rule{\linewidth}{0.5mm}} % Defines a new command for the horizontal lines, change thickness here

\center % Center everything on the page

\textsc{\LARGE Universidad de Salamanca}\\[1.5cm] % Name of your university/college
\textsc{\Large Administración de Sistemas}\\[0.5cm] % Major heading such as course name
%\textsc{\large Minor Heading}\\[0.5cm] % Minor heading such as course title

\HRule \\[0.4cm]
{ \huge \bfseries Python}\\[0.4cm] % Title of your document
\HRule \\[1.5cm]

\begin{minipage}{0.4\textwidth}
\begin{flushleft} \large
\emph{Autor:}\\
Diego \textsc{Martín} % Your name
\end{flushleft}
\end{minipage}
~
\begin{minipage}{0.4\textwidth}
\begin{flushright} \large
\emph{Profesora:} \\
Dr. Vivian \textsc{López} % Supervisor's Name
\end{flushright}
\end{minipage}\\[4cm]

%% Spanish!
{\large \today}\\[3cm] % Date, change the \today to a set date if you want to be precise

%\includegraphics{Logo}\\[1cm] % Include a department/university logo - this will require the graphicx package

\vfill % Fill the rest of the page with whitespace

\end{titlepage}

%----------------------------------------------------------------------------------------
%	TABLE OF CONTENTS
%----------------------------------------------------------------------------------------

\tableofcontents % Include a table of contents

\newpage % Begins the essay on a new page instead of on the same page as the table of contents 

%----------------------------------------------------------------------------------------
%	INTRODUCTION
%----------------------------------------------------------------------------------------

\large The Zen of Python\\
\small{Beautiful is better than ugly.\\
    Explicit is better than implicit.\\
    Simple is better than complex.\\
    Complex is better than complicated.\\
    Flat is better than nested.\\
    Sparse is better than dense.\\
    Readability counts.\\
    Special cases aren't special enough to break the rules.\\
    Although practicality beats purity.\\
    Errors should never pass silently.\\
    Unless explicitly silenced.\\
    In the face of ambiguity, refuse the temptation to guess.\\
    There should be one-- and preferably only one --obvious way to do it.\\
    Although that way may not be obvious at first unless you're Dutch.\\
    Now is better than never.\\
    Although never is often better than *right* now.\\
    If the implementation is hard to explain, it's a bad idea.\\
    If the implementation is easy to explain, it may be a good idea.\\
    Namespaces are one honking great idea -- let's do more of those!\\} 

\section{Introducción} % Major section

\subsection{\large ¿Por qué usar Python?}
	Para aquel que no aún no conozca las virtudes de este lenguaje, y qué ventajas proporciona para sus labores como programador, puede que al principio se muestre reacio a adoptar esta tecnología. Sin embargo, puede que estos principios de desarrollo hagan cambiar de opinión hasta al más acérrimo a otros lenguajes.
	\begin{itemize}
		\item Python es legible y coherente.\\
			`Legible, por tanto mantenible', es una de las premisas del lenguaje. Se aleja de lenguajes como Perl en ese aspecto. Además es un lenguaje orientado a objetos, lo cual aumenta su reusabilidad enormemente.\\
			Muchos dicen que Python \textit{encaja en tu cerebro}. Todo es comprensible cuando se dominan unos pocos conceptos clave. Además, el enfoque general es muy minimalista.
		\item Potenciar la productividad del programador\\
			Un código en Python ocupa 1/5 de su equivalente en C o Java. Además se elimina el proceso de compilación, dado que es un lenguaje interpretado\footnote{A lo largo de la historia del desarrollo de software se han dado situaciones en las que era difícil encontrar suficientes programadores para implementar ciertas aplicaciones. Actualmente los equipos se ven forzados a conseguir las mismas tareas con menos gente. En ambos escenarios, Python permite conseguir más con menos.}.
		\item Portabilidad\\
			Todos los programas en Python son multiplataforma \textit{per se} o requieren muy pocos cambios para trasladar un programa a otro sistema. Esto se aplica también a las interfaces gráficas de usuario.
		\item Bibliotecas de soporte\\
			Python cuenta con una gran cantidad de recursos de serie, permitiendo conseguir tareas de nivel de aplicación con la funcionalidad con la que se cuenta desde el momento en el que se instala. Además, existe una gran cantidad de código desarrollado por terceros y la posibilidad de crear módulos personales para solucionar tareas.
		\item Integración\\
			Python se integra con lenguajes y tecnologías como C, C++, Java, COM, Corba, .NET, SOAM, XML-RPC
		\item Disfrútalo\\
			Python es un lenguaje pensado para hacer al programador la tarea de desarrollar software mucho más amigable y entretenida.
	\end{itemize}
		
%------------------------------------------------

\subsection{Contras} % Sub-section

El principal problema de Python es su rendimiento. Al ser un lenguaje interpretado y no compilado, no es tan potente como C o C++. Sin embargo actualmente se ha mejorado notablemente su rendimiento gracias a la traducción del código a un \textit{bytecode}, que es realmente el conjunto de sentencias utilizado por la PVM (Python Virtual Machine).
Generalmente es un lenguaje rápido, y cuando es necesario optimizar el rendimiento al máximo, se puede realizar una traducción a C y traducirlo posteriormente a código máquina, por lo que se ejecutaría a velocidades nativas. También es posible escribir estas partes directamente en C y enlazarlas al resto del código.

%------------------------------------------------

\subsection{¿Quién usa Python?} % Sub-section

Con una comunidad de entre 500,000 y un millón de usuarios, Python cuenta con una comunidad alrededor muy grande. Compañías como Google o Yahoo! utilizan el lenguaje en sus servicios en red, y para HP, Seagate o IBM es una herramienta de utilidad a la hora de realizar pruebas de hardware\footnote{La lista completa se puede encontrar en \href{http://www.python.org}{python.org}}
Proyectos como el microordenador Raspberry Pi promocionan y utilizan Python como lenguaje principal de programación a la hora de enseñar a nuevos programadores y al utilizar el dispositivo.

%%------------------------------------------------
%
%\subsubsection{Subsubsection 1} % Sub-sub-section
%
%\lipsum[3] % Dummy text
%
%\begin{figure}[H] % Example image
%\center{\includegraphics[width=0.5\linewidth]{placeholder}}
%\caption{Example image.}
%\label{fig:speciation}
%\end{figure}
%
%%------------------------------------------------

%\subsubsection{Subsubsection 2} % Sub-sub-section
%
%\lipsum[4] % Dummy text

\subsection{¿Qué se hace con Python?}
\begin{description}
	\item[Administración de sistemas]
	Crear interfaces para servicios de un sistema operativo es muy sencillo en Python, lo cual lo convierte en una herramienta idónea para administradores de sistemas a la hora de trabajar con árboles de directorios, búsqueda de ficheros, lanzamiento de otros programas, procesamiento paralelo \dots
	\item[Interfaces Gráficas de Usuario]
	Crear interfaces en Python es muy simple. El entorno trabaja de forma natural con objetos diseñados para crear IGUs, como Tk Tkinter, que se adapta a la estética de cada plataforma, wxPython, basada en C++ o PythonCard.
	\item[Scripting]
	Es fácil realizar tareas de red con Python, y es muy utilizado para CGI(\textit{Common Gateway Interface}), FTP, sockets o HTML. Módulos como HTMLGen permiten generar código y otros como win32all insertan código Python como JavaScript. Además, utilizades como Zope, WebWare o Quixote permiten el desarrollo de sitios web grandes de forma rápida y sencilla.
	\item[Prototipado]
	Debido a que el desarrollo en Python es muy ágil, muchos desarrolladores aprovechan esta característica para generar `esbozos' de sus proyectos para evaluarlos antes de crear la versión definitiva en otro lenguaje. Aprovechando además la integración con otros lenguajes, en ocasiones partes del prototipo en Python son integradas en el resto del código.
	\item[Programación numérica, juegos, trabajo con imágenes, inteligencia artificial, etcétera]
	%%Revisar
\end{description}

\subsection{Fortalezas técnicas de Python}
\begin{itemize}
	\item Es un lenguaje orientado a objetos.
	\item Es libre (licencia de la Python Software Foundation, compatible con la GPL).
	\item Muy soportado.
	\item Espíritu de comunidad.
		Guido van Rossum, creador de Python es el \textit{Benevolent Dictator For Life of Python} (Benevolente dictador vitalicio de Python, orquesta a un equipo de 1000 personas encargados de mejorar el lenguaje. Los cambios siguen un proceso de mejora formal, que es analizado por Guido. Esta metodología hace que el desarrollo sea mucho más conservador que en otros lenguajes.
	\item Portabilidad
	\item Potencia
		\subitem Es un lenguaje dinámicamente tipado.
		\subitem Gestión automática de la memoria, utilizando un contador de referencias
		\subitem Tipos predefinidos que cubren la mayoría de las estructuras de datos utilizadas.
		\subitem Gran variedad de herramientas incluidas en el lenguaje.
		\subitem Gran cantidad de bibliotecas y código de terceros
	\item Se combina con facilidad con otros lenguajes
	\item Es fácil de utilizar y aprender.
	\item Sintaxis clara.
\end{itemize}
%----------------------------------------------------------------------------------------
%	MAJOR SECTION 1
%----------------------------------------------------------------------------------------

\subsection{¿Cómo se ejecuta un programa en Python?} % Major section

Para ejecutar un programa en Python es necesario únicamente un intérprete y una biblioteca de soporte. El código es compilado a \textit{bytecode}, un código de bajo nivel e independiente de la plataforma utilizada. Un archivo .pyc es generado, y es interpretado por la máquina virtual de Python del sistema\footnote{Hay varias implementaciones de la máquina virtual: CPython está escrita en ANSI C, Jython, que cuenta con clases de Java que compilan el código de Python a \textit{bytecode} de Java, dirigiéndolo a la \textit{Java Virtual Machine} del sistema o Python.NET, que se integra con C\#.}.
Existen además implementaciones especiales, como JIT Python, que incorpora un compilador \textit{Just In Time}, o binarios `congelados' (\textit{frozen binaries}), que empaquetan todo lo necesario para que un programa se ejecute (máquina virtual, código, etcétera).

En todo código se deberá indicar la ruta del intérprete. Utilizando el fichero \verb+/usr/bin/env python+ obtenemos la ruta se encuentre donde se encuentre el intérprete, por lo que evitamos modificar el código a la hora de cambiar de sistema.
	\lstset{language=python, showspaces=false}
	\begin{lstlisting}[frame=single, showspaces=false]
	#!/usr/bin/env python
	\end{lstlisting}

También es posible trabajar directamente con la línea de comandos, y el lenguaje es compatible con tuberías y redirección de las diferentes entradas y salidas.

%%------------------------------------------------
%
%\subsection{Subsection 1} % Sub-section
%
%\subsubsection{Subsubsection 1} % Sub-sub-section
%
%\lipsum[6] % Dummy text
%
%%------------------------------------------------
%
%\subsubsection{Subsubsection 2} % Sub-sub-section
%
%\lipsum[6] % Dummy text
%\begin{wrapfigure}{l}{0.4\textwidth} % Inline image example
%  \begin{center}
%    \includegraphics[width=0.38\textwidth]{fish}
%  \end{center}
%  \caption{Fish}
%\end{wrapfigure}
%\lipsum[7-8] % Dummy text
%
%%------------------------------------------------
%
%\subsubsection{Subsubsection 3} % Sub-sub-section
%
%\begin{description} % Numbered list example
%
%\item[First] \hfill \\
%\lipsum[9] % Dummy text
%
%\item[Second] \hfill \\
%\lipsum[10] % Dummy text
%
%\item[Third] \hfill \\
%\lipsum[11] % Dummy text
%
%\end{description} 

%----------------------------------------------------------------------------------------
%	MAJOR SECTION X - TEMPLATE - UNCOMMENT AND FILL IN
%----------------------------------------------------------------------------------------

%\section{Content Section}

%\subsection{Subsection 1} % Sub-section

% Content

%------------------------------------------------

%\subsection{Subsection 2} % Sub-section

% Content

%----------------------------------------------------------------------------------------
%	CONCLUSION
%----------------------------------------------------------------------------------------

\section{Números}

Trabajar con valores numéricos algo habitual en, si no la totalidad de las aplicaciones en cualquier lenguaje, en una inmensa mayoría. Python incorpora tipos predefinidos para simplificar las tareas más habituales.
Tipos de números:
\begin{itemize}
	\item Enteros (\verb+int+)
	\item Real en coma flotante (\verb+float+) y de doble precisión (\verb+float+)
	\item Como diferencia con otros lenguajes, Python soporta de forma nativa operaciones con números complejos.
\end{itemize}
El límite de tamaño para un número en Python lo determina la memoria, y debido al carácter dinámico del lenguaje, no es necesario definir el tipo de número. Operaciones con octales y hexadecimales son triviales también. La conversión es automática y se realiza siempre (a menos q	ue se especifique de forma explícita).
La representación es variable, utilizándose las funciones \verb+str()+ y \verb+repr()+ para cambiar de la forma utilizada por el ordenador a una representación más `humana'.

Los operadores de C han sido integrados en Python respetando la precedencia de cada uno. Se incluyen algunas novedades y mejoras en algunos operadores. Destacan:
\begin{itemize}
	\item Creación de funciones sin nombre (`anónimas'): \verb+lambda <argumentos>: <expresión>+
	\item \verb+is+ para determinar el tipo de objeto.
	\item Operador // para división con truncado.
	\item oct(), hex(), str(), repr\dots para conversiones de tipo.
	\item Nuevas operaciones con arrays y otras colecciones de datos (se verán más tarde)
\end{itemize}

\subsection{¿Cómo se almacena un elemento en memoria?}

Aprovechando que hemos cubierto el primer grupo de tipos de datos, es importante conocer cómo funciona una variable declarada en Python. Toda variable almacena una referencia a un elemento en memoria, por lo que las mismas no tienen un tipo definido. Esto no significa que los elementos en memoria no tengan tipos definidos. Al contrario, es importante conocer las características de cada uno, dado que la conversión entre tipos no es automática. Funciones como str() permiten realizar la transformación.
El contador de referencias es el encargado de la gestión de la memoria, liberando el espacio ocupado por un bloque de datos cuando se pierde la última referencia.
\section{Cadenas de texto}

Una cadena de texto en Python es un tipo especial de secuencia (las veremos posteriormente). No existe el tipo \verb+char+ como en C y, a diferencia de este, Python cuenta con un potente conjunto de utilidades para manipular cadenas.
Las cadenas de texto son elementos inmutables, por lo que una vez creados no se podrán modificar (es necesario crear una nueva cadena con los cambios si se desea alterar uno de estos elementos).
Ejemplos de literales \verb+string+:
\begin{itemize}
	\item Cadena vacía: \verb+cadena = ''+
	\item Cadena vacía (con dobles comillas): \verb+cadena = ""+
	\item Bloque vacío (un bloque permite representar cadenas de texto de varias líneas): \verb+bloque = """..."""+
	\item Cadena en bruto: \verb+bruto = r'\temp\haus'+
	\item Unicode: \verb+u'universal'+
	\item Formateo: \verb+"Resultados: %d" % resultados+
\end{itemize}

\subsection{Manipulación de cadenas}
\Large Operadores
\begin{itemize}
	\item Concatenación: \verb|+|
	\item Repetición: \verb+*+
	\item Índice: \verb+cadena[indice]+
	\item Longitud: \verb+len(cadena)+
	\item Búsqueda de elementos: \verb+cadena.find('Waldo')+
	\item Iteración: \verb+for x in cadena+
	\item Longitud: \verb+len()+
\end{itemize}

\Large Métodos
\begin{itemize}
%%Completar
	\item \verb+cadena.capitalize()+
	\item \verb+cadena.center()+
	\item \verb+.encode(), .endswith(), .expandtabs()+ \dots
	\item \verb+.isalnum(), .isdigit(), .istitle(), .isupper()+ \dots
\end{itemize}


El soporte de Unicode está integrado en el lenguaje, y siempre que se combinen con otros tipos de cadenas (por ejemplo, en una concatenación) el resultado será una cadena Unicode.
Dado que una cadena es un conjunto ordenado de caracteres pueden ser manipuladas mediante índices.

\subsection{Secuencias de escape}
\begin{itemize}
	\item \verb+\newline+
%completar
\end{itemize}

\subsection{Formateo avanzado de cadenas}
Se puede utilizar cualquier especificados de formato de C en Python, así como alguno nuevo.
\begin{itemize}
	\item \verb+%s+
	\item \verb+%d+
	\item \verb+%r+
\end{itemize}

\subsection{Consideraciones especiales}
Una cadena no se convierte automáticamente a número o a cualquier otro tipo de datos, es necesario hacerlo explícitamente.

\section{Listas y diccionarios}
Existen tres categorías generales en Python: números, secuencias y mapas. Cada una de ellas tiene unas propiedades especiales y operaciones asociadas a él. Hasta ahora hemos visto las dos primeras (números y secuencias, dado que las cadenas de caracteres son un tipo de estas). La categoría restante, los mapas

%\lipsum[12-13]

%----------------------------------------------------------------------------------------
%	BIBLIOGRAPHY
%----------------------------------------------------------------------------------------

%%Use
\begin{thebibliography}{99} % Bibliography - this is intentionally simple in this template

\bibitem[The Zen of Python, 2004]{Peters:2004dg}
Figueredo, A.~J. and Wolf, P. S.~A. (2009).
\newblock Assortative pairing and life history strategy - a cross-cultural
  study.
\newblock {\em Human Nature}, 20:317--330.
 
\end{thebibliography}

%----------------------------------------------------------------------------------------

\end{document}